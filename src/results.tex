\section{Results \& Evaluation}
We evaluate our implementation based on two metrics: \textit{correctness} and \textit{performance}.
Correctness relates to the functionality of the filesystem, i.e. to what degree the functionality of the implementation conforms to its specification.
Performance describes the speed and efficiency of the implementation.

\subsection{Correctness}
As stated earlier, the filesystem was split into two parts: a filesystem logic part (the \textit{adafs} package), and an I/O part (the \textit{disk} package).
Because the logic part is easier to verify, it was the primary focus of formal verification for this project.
Currently, approximately 25\% of the filesystem is verified, and most of the unverified code is in the \textit{disk} package.
Since the FUSE driver is written in C, and in Ada with C types, it is also not verified.

Subprograms were given functional and data contracts, and type definitions according to their expected functionality.
Although this does not formally verify the functionality of the entire filesystem, it is a good starting point.
\autoref{code:prover summary} shows a summary of the conducted checks, and \autoref{code:prover output} shows some of the locations of various checks.

\begin{lstlisting}[float=tb,basicstyle=\tiny, caption={Prover summary (\textnormal{gnatprove})}, label={code:prover summary}]
Summary of SPARK analysis
=========================

----------------------------------------------------------------------------------------------------------
SPARK Analysis results   Total        Flow   CodePeer                       Provers   Justified   Unproved
----------------------------------------------------------------------------------------------------------
Data Dependencies            5           5          .                             .           .          .
Flow Dependencies            4           4          .                             .           .          .
Initialization               1           1          .                             .           .          .
Non-Aliasing                 .           .          .                             .           .          .
Run-time Checks             17           .          .    17 (CVC4 89%, Trivial 11%)           .          .
Assertions                   2           .          .                      2 (CVC4)           .          .
Functional Contracts         6           .          .      6 (CVC4 92%, Trivial 8%)           .          .
LSP Verification             .           .          .                             .           .          .
Termination                  .           .          .                             .           .          .
Concurrency                  .           .          .                             .           .          .
----------------------------------------------------------------------------------------------------------
Total                       35    10 (29%)          .                      25 (71%)           .          .

max steps used for successful proof: 1

Analyzed 12 units
\end{lstlisting}

\begin{lstlisting}[float=tb,basicstyle=\tiny, caption={Excerpt from the prover output (\textnormal{gnatprove})}, label={code:prover output}]
adafs-inode.adb:6:27: info: division check proved (CVC4: 1 VC in max 0.0 seconds and 1 step)
adafs-inode.adb:6:27: info: range check proved (CVC4: 1 VC in max 0.0 seconds and 1 step)
adafs-inode.adb:8:44: info: division check proved (CVC4: 1 VC in max 0.0 seconds and 1 step)
adafs-inode.adb:13:12: info: overflow check proved (CVC4: 1 VC in max 0.0 seconds and 1 step)
adafs-inode.adb:14:12: info: division check proved (CVC4: 1 VC in max 0.0 seconds and 1 step)
adafs-inode.adb:14:12: info: range check proved (CVC4: 1 VC in max 0.0 seconds and 1 step)
adafs-inode.adb:14:28: info: overflow check proved (CVC4: 1 VC in max 0.0 seconds and 1 step)
adafs-inode.ads:40:60: info: range check proved (CVC4: 1 VC in max 0.0 seconds and 1 step)
adafs-inode.ads:66:10: info: data dependencies proved
adafs-inode.ads:67:10: info: flow dependencies proved
adafs-inode.ads:69:18: info: postcondition proved (CVC4: 2 VC in max 0.0 seconds and 1 step)
adafs-proc.adb:6:14: info: discriminant check proved (CVC4: 2 VC in max 0.0 seconds and 1 step)
adafs-proc.adb:12:16: info: discriminant check proved (CVC4: 2 VC in max 0.0 seconds and 1 step)
adafs-proc.ads:26:05: info: data dependencies proved
adafs-filp.adb:18:30: info: loop invariant initialization proved (CVC4: 1 VC in max 0.0 seconds and 1 step)
adafs-filp.adb:18:30: info: loop invariant preservation proved (CVC4: 1 VC in max 0.0 seconds and 1 step)
adafs-filp.ads:28:13: info: postcondition proved (CVC4: 3 VC in max 0.0 seconds and 1 step)
adafs-filp.ads:32:28: info: initialization of "free_fd" proved
adafs-filp.ads:33:05: info: data dependencies proved
adafs-filp.ads:34:05: info: flow dependencies proved
adafs-filp.ads:35:13: info: postcondition proved (CVC4: 2 VC in max 0.0 seconds and 1 step)
\end{lstlisting}

The checks done by the prover can each be related directly to a Common Weakness Enumeration (CWE) number; these are shown in \autoref{tab:checks and cwe numbers} \cite{sparkUG}.

\begin{table}[tb]
  \centering
  \vspace{1em}
  \renewcommand{\arraystretch}{1.5}
  \begin{tabular}{| l | l |}
    \hline
    \textbf{Message} & \textbf{CWE number} \\ \hline
    divide by zero & 369 \\ \hline
    index check & 120 \\ \hline
    overflow check & 190 \\ \hline
    fp\_overflow check & 739 \\ \hline
    range check & 682 \\ \hline
    predicate check & 682 \\ \hline
    predicate check on default value & 682 \\ \hline
    null pointer dereference & 476 \\ \hline
    memory leak & 401 \\ \hline
    memory leak at end of scope & 401 \\ \hline
    discriminant check & 136 \\ \hline
    tag check & 136 \\ \hline
    use of an uninitialized variable & 457 \\ \hline
    precondition & 628 \\ \hline
    postcondition & 682 \\ \hline
  \end{tabular}
  \renewcommand{\arraystretch}{1}
  \vspace{1em}
  \caption{SPARK checks with their respective CWE numbers}
  \label{tab:checks and cwe numbers}
\end{table}

Due to Ada's type system, many of the checks do not take much effort to verify; in fact, all of the conducted checks were verified in one step.
The assurances provided by the prover are powerful, as they mean that e.g. overflow errors and range errors can \textit{never} happen at those locations.
In standard Ada, instructions are emitted to check that e.g. array accesses are within the bounds of the array.
However, with SPARK, such verifications can be conducted ahead of time.
This means that the runtime checks done by Ada can be safely disabled, eliminating a number of machine instructions and thus improving the performance of the system.

The next step in verification would be to verify filesystem consistency, including consistency in the face of crashes.
This verification was not conducted in this paper, as the packages used for disk IO are not written in the SPARK subset of Ada, and are therefore not amenable to formal verification.
However, it would be possible to implement a custom IO package that is written in SPARK, and can therefore be verified.
An alternative approach could be to verify the disk IO component using alternative methods, for example a modified version of Hoare Logic discussed in \autoref{sec:related work}, and this component would export several functions to the Ada filesystem.
A third option could be to implement a verified transactional log.

\subsection{Performance}
We conduct several tests to compare the performance of AdaFS with its C counterpart.
In particular, we would like to find out if the extra formal verification code results in decreased performance, as well as the effect that interfacing between C and Ada has on system performance.

\subsubsection{Microbenchmark comparison of Ada and C filesystem implementations}
We compare the performance of AdaFS with a C implementation of a FUSE-based MINIX 3 filesystem, fuse-mfs.\footnote{\url{https://github.com/redcap97/fuse-mfs}}
As AdaFS is not a complete filesystem, but only supports a small set of operations, it was not possible to use a standardised utility to test it.
Therefore, we created a microbenchmark in C that executes the relevant system calls and counts the number of CPU clock cycles necessary for each operation.
We compared some common filesystem operations, such as creating, writing, reading, and deleting files.
We used a virtual machine in VirtualBox 6.1.10 on a macOS host, running Debian 4.19.118, with FUSE version 3.4.1.
The host computer has an Intel Core i7-3615QM processor, with 4 physical cores and two threads per core, at 2.3 GHz.
The virtual machine is assigned 4 of these 8 virtual processors.
The disk images for the filesystem are stored on an SSD.

To calculate the number of CPU clock cycles an operation takes, we use a deterministic version of the ReaD Time Stamp Counter (RDTSC) instruction, RDTSCP, which is added as inline assembly in the C microbenchmark code.
The code to calculate elapsed clock cycles is shown in \autoref{code:rdtsc}.

\begin{lstlisting}[float=tb,caption={Calculating clock cycles}, label={code:rdtsc}, language=C]
static __inline__ int64_t rdtsc_s(void) {
  unsigned a, d;
  asm volatile("cpuid" ::: "%rax", "%rbx", "%rcx", "%rdx");
  asm volatile("rdtsc" : "=a" (a), "=d" (d));
  return ((unsigned long)a) | (((unsigned long)d) << 32);
}

static __inline__ int64_t rdtsc_e(void) {
  unsigned a, d;
  asm volatile("rdtscp" : "=a" (a), "=d" (d));
  asm volatile("cpuid" ::: "%rax", "%rbx", "%rcx", "%rdx");
  return ((unsigned long)a) | (((unsigned long)d) << 32);
}
typedef int64_t cycle_t;

int main() {
  cycle_t cycles_before, cycles_after, cycles_per;
  cycles_before = rdtsc_s();
  // Call the test function here...
  cycles_after = rdtsc_e();
  cycles_per = cycles_after-cycles_before;
  printf("CPU cycles: %ld\n", cycles_per);
}
\end{lstlisting}

We choose RDTSCP instead of RDTSC because the CPU cannot reorder it relative to other instructions, as such reordering would cause noise in our results \cite{delorie2016}.
We set the \lstinline{isolcpus} kernel option at boot to isolate CPU 1, and then use the \lstinline{sched_affinity} function call to explicitly schedule each test on this CPU.
\autoref{code:schedule test} shows the code for this.

\begin{lstlisting}[float=tb,caption={Schedule a test on CPU 1}, label={code:schedule test}, language=C]
#define _GNU_SOURCE
#include <sched.h>
void run_test(void) {
  // Schedule
  cpu_set_t the_cpu;
  int the_cpu_num = 1;
  CPU_ZERO(&the_cpu);
  CPU_SET(the_cpu_num, &the_cpu);
  if (sched_setaffinity(0, sizeof(the_cpu), &the_cpu) == -1) {
    perror("setaffinity failed");
  }

  // Run any tests below
  ...
}
\end{lstlisting}

We also assign interrupts to CPU 0 to avoid interference.
This ensures that no other processes are scheduled on the same CPU as our tests, i.e. that our results are as noiseless as possible.
We run FUSE in single-thread mode for both implementations of the filesystem.

We run the benchmark 30 times for each filesystem implementation; \autoref{tab:fs comparison} presents the median values for each implementation and operation.
It is apparent that AdaFS is significantly slower than fuse-mfs.
The largest difference is apparent in the create operation, which is approximately 412.5 times slower.
The remove operation is around 2.5 times slower than that of fuse-mfs.
For other operations, the difference is not as drastic, though the Ada implementation is still around 10-20\% slower on average.

\begin{table}[tb]
  \begin{subtable}[t]{\textwidth}
    \centering
    \begin{tabular}{l r | r | r | r}
                      & \multicolumn{2}{c}{\underline{Create}}                                           & \multicolumn{2}{c}{\underline{Remove}} \\
       Implementation & Cycles                                 & Std. dev.                               & Cycles                   & Std. dev.                   \\
      \hline \hline
      AdaFS           & \result{adafs-create}                  & \result{adafs-create-stdev}             & \result{adafs-remove}    & \result{adafs-remove-stdev} \\
      fuse-mfs        & \result{fuse-mfs-create}               & \result{fuse-mfs-create-stdev}          & \result{fuse-mfs-remove} & \result{fuse-mfs-remove-stdev}
    \end{tabular}
    \caption{CPU cycles to create and remove files}
    \label{tab:create remove files}
  \end{subtable}

  \bigskip

  \begin{subtable}[t]{\textwidth}
    \centering
    \small
    \begin{tabular}{l r | r | r | r | r | r | r | r}
                     & \multicolumn{2}{c}{\underline{Read 1 B}} & \multicolumn{2}{c}{\underline{Read 1 KB}} & \multicolumn{2}{c}{\underline{Read 10 KB}} & \multicolumn{2}{c}{\underline{Read 100 KB}} \\
      Implementation & Cycles                   & Std. dev.                      & Cycles                      & Std. dev.                         & Cycles                       & Std. dev.                          & Cycles                        & Std. dev.                        \\
      \hline \hline
      AdaFS          & \result{adafs-read-1}    & \result{adafs-read-1-stdev}    & \result{adafs-read-1024}    & \result{adafs-read-1024-stdev}    & \result{adafs-read-10240}    & \result{adafs-read-10240-stdev}    & \result{adafs-read-102400}    & \result{adafs-read-102400-stdev} \\
      fuse-mfs       & \result{fuse-mfs-read-1} & \result{fuse-mfs-read-1-stdev} & \result{fuse-mfs-read-1024} & \result{fuse-mfs-read-1024-stdev} & \result{fuse-mfs-read-10240} & \result{fuse-mfs-read-10240-stdev} & \result{fuse-mfs-read-102400} & \result{fuse-mfs-read-102400-stdev}
    \end{tabular}
    \caption{CPU cycles to read files of varying sizes}
    \label{tab:read files}
  \end{subtable}

  \bigskip

  \begin{subtable}[t]{\textwidth}
    \centering
    \small
    \begin{tabular}{l r | r | r | r | r | r | r | r}
                     & \multicolumn{2}{c}{\underline{Write 1 B}} & \multicolumn{2}{c}{\underline{Write 1 KB}} & \multicolumn{2}{c}{\underline{Write 10 KB}} & \multicolumn{2}{c}{\underline{Write 100 KB}} \\
      Implementation & Cycles                                  & Std. dev.                                & Cycles                                    & Std. dev.                                   & Cycles                       & Std. dev.                          & Cycles                        & Std. dev.                        \\
      \hline \hline
      AdaFS          & \result{adafs-write-1}                   & \result{adafs-write-1-stdev}              & \result{adafs-write-1024}                  & \result{adafs-write-1024-stdev}              & \result{adafs-write-10240}    & \result{adafs-write-10240-stdev}    & \result{adafs-write-102400}    & \result{adafs-write-102400-stdev} \\
      fuse-mfs       & \result{fuse-mfs-write-1}                & \result{fuse-mfs-write-1-stdev}           & \result{fuse-mfs-write-1024}               & \result{fuse-mfs-write-1024-stdev}           & \result{fuse-mfs-write-10240} & \result{fuse-mfs-write-10240-stdev} & \result{fuse-mfs-write-102400} & \result{fuse-mfs-write-102400-stdev}
    \end{tabular}
    \caption{CPU cycles to write files of varying sizes}
    \label{tab:write files}
  \end{subtable}
  \caption{Comparing filesystem operation performance in C (fuse-mfs) and Ada (AdaFS), showing the median and standard deviation across 30 observations.}
  \label{tab:fs comparison}
\end{table}

These differences in performance are mostly small enough that they are not generally noticeable in everyday life.
However, a create operation that takes 0.24 seconds (548615020 cycles at 2.3 GHz) may have a visible impact, and the other operations may also cause issues in applications where speed is critical.
Thus, it is important to find out why this slowdown happens, or eliminate factors that do not cause decreased performance.
It may not necessarily be due to the extra code needed for formal verification; AdaFS does not employ any optimisations or caching that MINIX 2 uses, and MINIX 3 is even more optimised for speed than MINIX 2.
For example, part of the \textit{create} operation is the creation of a new inode.
Among other things, MINIX keeps track of free inodes in a table in memory, which is flushed to disk at various points.
This makes looking up a free inode much faster, improving the performance of the \textit{create} operation.
Due to time constraints, AdaFS does not implement this table, and thus has to read it from the disk.
This is also the case with other similar mechanisms.

\subsubsection{The performance impact of formal verification code}
To determine if the verification code may play a role in the slowdown, we implemented two versions of a factorial function for numbers up to and including 20, in Ada.
We chose 20 because this is the highest possible number whose factorial is within the allowed range; Ada has support for larger numbers via various packages, but we opted not to use them so as not to introduce unnecessary overhead.
We use the same settings and pragmas as AdaFS.
One version includes a full formal specification, with data and functional contracts, and with type definitions.
The other version does not have any verification code whatsoever.
Both functions are exported to, and called from, C code; there is no parameter type conversion.

We run both implementations 500 times, counting the clock cycles.
The results (median and standard deviation) are shown in \autoref{tab:factorial comparison}.
From this, we see that the verified code actually seems to be faster.
However, the difference is only 4 clock cycles (1.2\%), so though verifying code with SPARK may allow the compiler to omit some instructions to check values and ranges, it is more probable that the difference is the result of some unknown interference.
For practical purposes, there does not seem to be a performance difference between code with formal verification annotations, and code without them.

\begin{table}[tb]
  \centering
  \begin{tabular}{l | r | r}
    Implementation & Cycles                         & Std. dev.                         \\
    \hline \hline
    Verified       & \result{factorial-verified}    & \result{factorial-verified-stdev} \\
    Unverified     & \result{factorial-nonverified} & \result{factorial-nonverified-stdev}
  \end{tabular}
  \caption{Factorial function comparison in Ada and C.}
  \label{tab:factorial comparison}
\end{table}

\subsubsection{The performance impact of C-to-Ada interfacing}
There is also another variable that must be accounted for: whether a function is called from C or from Ada.
The operations in AdaFS are called from C, with C types.
The C types passed into the function need to be converted to their corresponding Ada types, and then converted back when they are returned.
To see if some extra clock cycles might be used at this boundary, we implemented a function that takes an integer, multiplies it by two, and returns the result.
In C, this function is trivial; in Ada, the function converts the argument from the C integer type into Ada's Integer type, and then converts it back to a C integer when returning from the function.

We call both implementations from C code, and run them 500 times each, counting the clock cycles.
The median and standard deviation for both implementations are shown in \autoref{tab:interface comparison}.
We see that calling Ada code from C code adds 8 extra clock cycles (i.e. calling a function from Ada is 10\% slower).
This also probably varies depending on the number of parameter conversions that are necessary.

\begin{table}[tb]
  \centering
  \begin{tabular}{l | r | r}
    Implementation & Cycles & Std. dev. \\
    \hline \hline
    Ada & \result{intf-ada} & \result{intf-ada-stdev} \\
    C & \result{intf-c} & \result{intf-c-stdev}
  \end{tabular}
  \caption{The performance impact of interfacing with Ada.}
  \label{tab:interface comparison}
\end{table}

\subsubsection{Comparing the performance of random-access IO packages}
Another potential source of performance degradation could be the choice of IO package.
Ada offers two packages for random-access data IO: \textit{Stream\_IO} (heterogeneous data) and \textit{Direct\_IO} (homogeneous data).
In this project, we opted for the former, but the latter may offer better performance.
For example, because the implementation of \textit{Stream\_IO} might include some code to manipulate incoming or outgoing data into a suitable format.
To test whether this is the case, we implemented a function that opens a file, reads a record from the file, modifies the record, writes the modified record to the file, and closes the file.
We created two implementations, one with \textit{Direct\_IO}, and one with \textit{Stream\_IO}.
Both implementations exported functions to C, and both were called from C; this helped eliminate any potential clock cycle differences at the boundary between C code and Ada code.
We used the same setup as for the other tests (i.e. schedule the tests on an isolated CPU, and count clock cycles with RDTSC), and ran each implementation's function 500 times.
The results (median and standard deviation) are shown in table \autoref{tab:io implementation comparison}.

\begin{table}[tb]
  \centering
  \begin{tabular}{l | r | r}
    Implementation & Cycles & Std. dev. \\
    \hline \hline
    Stream IO & \result{stream-io} & \result{stream-io-stdev} \\
    Direct IO & \result{direct-io} & \result{direct-io-stdev}
  \end{tabular}
  \caption{Random-access IO implementation comparison in Ada.}
  \label{tab:io implementation comparison}
\end{table}

We can see that \textit{Stream IO} is actually faster than Direct IO, taking around 25\% fewer clock cycles for the same operations.
One reason might be because Direct IO could have some checks in place to verify that the data is indeed homogeneous.
The difference in clock cycles could also be the result of switching between reading and writing modes.
\textit{Stream IO} has a dedicated function to set the mode of the file, which may be optimised in some way, while with Direct IO, the file has to be closed and reopened with the new mode.
Therefore, for the purposes of AdaFS where switches between file modes are common, \textit{Stream IO} seems more performant.
It could also be worthwhile to compare \textit{Stream IO} with one of the sequential IO packages, but as using only sequential reads and writes would require a different filesystem design or some levels of abstraction, it is not covered in this paper.

\subsubsection{Summary of performance evaluation}
Overall, there is quite a large difference between AdaFS and the C implementation of MINIX 3, with AdaFS being around 10-20\% slower on average, and the create operation being 412.5 times slower.
From follow-up testing, we found that interfacing between C and Ada adds 8 extra clock cycles, and that adding formal verification code to implementation code does not seem to impact performance.
We also compared two random-access IO packages, and found that using a homogeneous IO package does not seem to offer performance advantages over the heterogeneous IO package that we are currently using.
In summary, the performance difference we observe between AdaFS and its C counterpart is probably either due to optimisation differences, or due to some other variable that we have not tested.
