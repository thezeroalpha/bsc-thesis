\section{Introduction}
As K. J. Parker said, ``the fastest, cheapest and easiest way to build something is properly the first time'' \cite{parker2007}.
Software bugs can cost companies customers, reputation, and up to millions of dollars.
If these bugs are in the filesystem, they can destroy potentially irreplaceable and priceless data.
Unfortunately, given that large filesystem projects contain millions of lines of code, bugs are inevitable.
In 2017, a bug in the NT File System used by Windows was found, which allowed anyone to crash Windows 7 or 8.1 \cite{bright2017}.
Just this year, a bug in the Apple File System could prevent users from making a bootable clone of their disk \cite{bombich2020}.
There are currently 112 bugs reported in the Bugzilla database for the Ext4 filesystem, and 611 bugs reported for the Btrfs filesystem \cite{bugzilla2020}.

It is safe to say that we need a way to build more reliable software, without spending time and money fixing issues that could have been prevented from the start.
C is in widespread use in the development of operating systems and their components, including the Ext4 filesystem.
However, C is an inherently unsafe language, and its permissiveness means that errors are relatively easy to make.
Attacks exploiting these errors can be devastating, such as the 2001 CodeRed worm that infiltrated enterprise networks \cite{trendmicro2002}.

This paper explores alternative approaches to writing reliable software, specifically in the context of filesystems.
A small filesystem based on that of MINIX 2 is developed, in Ada and its SPARK subset.
Formal verification of some of its parts is conducted, and its effectiveness is analyzed and evaluated.
