\documentclass[12pt,notitlepage,]{article}
\usepackage[a4paper, top=0.7in, left=0.7in, right=0.7in, bottom=0.7in]{geometry}
\usepackage[utf8]{inputenc}
\usepackage[english]{babel}
\usepackage{hyperref}
\usepackage{xcolor}
\hypersetup{
    colorlinks,
    linkcolor={red!50!black},
    citecolor={blue!50!black},
    urlcolor={blue!80!black}
}
\usepackage{csquotes}
\usepackage{amsmath}
\usepackage{amssymb}
\usepackage{caption}
\usepackage{subcaption}
\captionsetup{width=0.8\textwidth,font+=it}

\usepackage{listings}
\lstset{captionpos=b, frame=lines, xleftmargin=1em, framextopmargin=0.5em, framexbottommargin=0.5em, aboveskip=1em, belowskip=1em, linewidth=0.98\linewidth, basicstyle=\small}
\captionsetup[lstlisting]{justification=centering}

\usepackage{graphicx}
\graphicspath{{images/}{diagrams/}{../evaluation/plots/}}

\usepackage{import}
\usepackage{xifthen}
\usepackage{pdfpages}
\usepackage{transparent}
\newcommand{\incfig}[1]{%
  \def\svgwidth{\columnwidth}
  \resizebox{0.85\textwidth}{!}{\import{./images/}{#1.pdf_tex}}
}
\newcommand{\result}[1]{\input{../evaluation/stats/#1}\unskip}

\usepackage[style=ieee]{biblatex}
\addbibresource{references.bib}

% To show overfull boxes:
\setlength{\overfullrule}{5pt}

\def\theauthor{Alexander Balgav\'{y} (2619644)}
\def\thetitle{AdaFS: An Exploration of the Use of Formal Methods For More Reliable Filesystems}
\def\thedate{\today}
\def\theinstitution{Vrije Universiteit Amsterdam, \\ Systems and Network Security Group (VUSec)}
\def\thesubject{Bachelor Thesis}

\begin{document}
  \begin{titlepage}
    \newcommand{\HRule}{\rule{0.8\linewidth}{0.2mm}}

    \centering

    \vspace*{1em}
    \textsc{\large \theinstitution}\\[1em]

    \includegraphics[width=0.45\textwidth]{vrije-universiteit-amsterdam.png} \\[2em]
    \includegraphics[width=0.4\textwidth]{vusec.png}

    \vspace{4em}
    \textsc{\Large \thesubject}\\
    \vspace{4em}

    \HRule\\[0.7cm]

    \begin{minipage}{0.75\textwidth}
      \centering
      \setlength{\baselineskip}{2em}
      {\LARGE\bfseries \thetitle}\\[1em]
      \vspace{1em}
    \end{minipage}

    \HRule\\[1.5cm]

    {\Large \theauthor}\\
    \vspace{2em}
    \begin{minipage}{0.72\textwidth}
      \large
      \centering
      % TODO: add supervisor/reader names
      \begin{tabular}{ r l }
        \textit{Supervisor:}        & Herbert Bos \\
        \textit{Daily supervisors:} & Sebastian \"{O}sterlund \& Hany Ragab \\
        \textit{Second reader:}     & Erik van der Kouwe
      \end{tabular}
    \end{minipage}

    \vfill
    \begin{minipage}{0.8\textwidth}
      \centering
      \textit{\large
        A thesis submitted in fulfillment of the requirements for the VU Bachelor of Science degree in Computer Science.
      }
    \end{minipage}

    \vspace{2em}
    {\large\today}

    \vspace{4em}
\end{titlepage}


  \begin{abstract}
    Unreliable filesystems have many dangerous implications, worst of all the loss of data.
    A well-designed filesystem is able to deal with errors by isolating them, repairing broken metadata, and mitigating damage.
    But could we instead prevent errors from occurring in the first place?
    In this paper, we take a look at current research done in the area of formal verification of filesystems.
    We develop a small filesystem in Ada, a language used for high-reliability software in a range of applications.
    We then evaluate it, finding that the language and its SPARK subset offer many powerful features that do not negatively impact the filesystem's performance.
  \end{abstract}
  \input{meta}
  \section{Introduction}
As K. J. Parker said, ``the fastest, cheapest and easiest way to build something is properly the first time'' \cite{parker2007}.
Software bugs can cost companies customers, reputation, and up to millions of dollars.
If these bugs are in the filesystem, they can destroy potentially irreplaceable and priceless data.
Unfortunately, given that large filesystem projects contain millions of lines of code, bugs are inevitable.
In 2017, a bug in the NT File System used by Windows was found, which allowed anyone to crash Windows 7 or 8.1 \cite{bright2017}.
Just this year, a bug in the Apple File System could prevent users from making a bootable clone of their disk \cite{bombich2020}.
There are currently 112 bugs reported in the Bugzilla database for the Ext4 filesystem, and 611 bugs reported for the Btrfs filesystem \cite{bugzilla2020}.

It is safe to say that we need a way to build more reliable software, without spending time and money fixing issues that could have been prevented from the start.
C is in widespread use in the development of operating systems and their components, including the Ext4 filesystem.
However, C is an inherently unsafe language, and its permissiveness means that errors are relatively easy to make.
Attacks exploiting these errors can be devastating, such as the 2001 CodeRed worm that infiltrated enterprise networks \cite{trendmicro2002}.

There have been efforts in the past to improve the reliability of systems, with the largest amount focusing on operating systems and their kernels.
There has also been some research in the area of filesystem reliability, and some projects have tried to develop frameworks for building reliable filesystems.
However, many of the frameworks require learning a language or programming style that is not very familiar to people who are used to working with C-style programming languages; moreover, some approaches may require radically changing the logic of the filesystem.

This paper explores an alternative way to write reliable software, particularly in the context of filesystems.
We use the Ada programming language and its SPARK subset, which were created specifically for the purpose of building high-reliability software.
We develop a small filesystem based on that of MINIX 2, demonstrating that using Ada could be a suitable approach for writing low-level filesystem code.
We conduct formal verification of some of the parts of the filesystem, and analyze and evaluate its effectiveness.

In short, this paper's main contributions are:

\begin{itemize}
  \item a novel approach to writing filesystems, using Ada, a statically-typed imperative language that is more similar to the traditional choice of C,
  \item formal verification of some parts of the filesystem, which are written in the SPARK subset of Ada and verified using predicate logic assertions, proving automatically that the code is free of errors,
  \item an evaluation of the performance effects of using Ada and conducting formal verification, compared to a C implementation.
\end{itemize}

  \section{Background information}
\subsection{C as the implementation language}
The choice of an implementation language may affect the bugs or vulnerabilities that are present in the system.
The de-facto standard implementation language for operating systems and their components has long been C.
Many popular file systems are implemented in C, such as Ext4. \cite{ext4source}.
C is based on typeless languages, BCPL and B, which were developed specifically for operating system programming in early Unix.
A design principle of C was to be grounded in the operations and data types provided by the computer, while offering abstractions and portability to the programmer. \cite{ritchie1993}

Describe studies on issues with C, bugs found, CVEs, etc.

\subsection{Possible alternatives}
\begin{itemize}
  \item Restricting C: MISRA-C, Frama-C
  \item Rust: esp. reference ownership
  \item D: allows functional contracts
  \item Coq: allows writing formal specification, proving it, and extracting certified program from constructive proof of its specification in OCaml, Haskell, or Scheme.
  \item Ada
\end{itemize}

\subsection{Formal verification}
Explain what it is, particularly Hoare triples.
Why is it useful?

\subsection{FUSE}
Why use it for development, what are its limitations?

  \section{Filesystem Design}
To explore ways of writing reliable code, a small, prototype filesystem (AdaFS) was implemented using the Ada 2012 programming language.
Some parts of this filesystem are written in the SPARK 2014 language, which is a subset of Ada that removes features not amenable to formal verification, and defines new aspects to support modular, constructive, formal verification \cite{sparkRM}.
The AdaCore GNAT Community 2020 package\footnote{\url{https://www.adacore.com/community}} is used, which provides, among others, the compiler and prover tools.
For testing purposes, a FUSE driver is written in C, and the built executables are linked with libfuse 3.9.2\footnote{\url{https://github.com/libfuse/libfuse}}.
The GNAT Project Manager is used to facilitate compilation of source files in different languages and linking of other required libraries.

AdaFS is based on the MINIX 2 filesystem \cite{tanenbaum1997}, with some simplifications due to time constraints.
The MINIX operating system was written by Andrew Tanenbaum as an educational tool, and is compatible with UNIX, but has a more modular structure.
The MINIX filesystem was chosen as a model because it is not part of the operating system, but rather runs entirely as a user program.
As such, it is self-contained.
Furthermore, due to its educational purpose, it is thoroughly commented and easy to understand.
The second version of the file system was chosen because MINIX 3 is more complex, as it adds numerous improvements (e.g. for reliability) to place emphasis on its use in research and production \cite{minix3history}.

A disk is formatted as an AdaFS filesystem using the \textit{mkfs} executable.
The resulting disk layout is shown in \autoref{fig:adafs disk layout}.
The disk is divided into blocks of 1024 bytes, similarly to MINIX.
Blocks are collected in zones, which can be of size $2^n$ blocks.
This abstraction of blocks into zones can make it possible to allocate multiple blocks at once.

\begin{figure}[tb]
  \centering
  \incfig{disk-layout}
  \caption{AdaFS disk layout. (\textit{n} = number of inode bitmap blocks, \textit{m} = number of zone bitmap blocks, \textit{i} = number of inode blocks, \textit{s} = number of blocks on disk)}
  \label{fig:adafs disk layout}
\end{figure}

The disk begins with a boot block that would optionally contain executable code.
Then, it contains a superblock, and two bitmaps.
The bitmaps are used for inode and zone allocation, and can potentially span multiple blocks.
Next, there are a few blocks containing space for inodes, potentially with more than one inode per block.
Finally, the rest of the blocks contain user data.

The superblock is the second block on the disk, and contains information about the layout of the filesystem.
In particular, it contains the number of inodes and zones on disk, the number of inode and zone bitmap blocks, the number of the first data zone, the base-2 logarithm of the number of blocks per zone, the maximum file size, and the magic number.
The magic number used to identify a correctly formatted disk is $\text{CACA}_{16}$; MINIX uses the magic number $2468_{16}$, but AdaFS avoids using the same number because a disk formatted by the AdaFS \textit{mkfs} utility is not necessarily equivalent to a disk formatted by the MINIX \textit{mkfs} utility.

The two bitmaps keep track of available inodes and zones, where the \textit{n}th bit of the inode or zone bitmap corresponds to the \textit{n}th inode or zone on disk, respectively.
If a bit in the inode bitmap is set to 1, that means the corresponding inode is allocated, and if it is set to 0, the inode is free.
This is the same for the zone bitmap, pertaining to zones.
One difference with MINIX is that, in MINIX, the first bit (bit zero) in the bitmaps must always be allocated, as the procedure that searches for a free inode or zone returns zero if no free inode/zone is found.
In Ada, indexing generally starts at 1, so all bits can be used -- as the first bit is bit one, this does not interfere with the allocation procedure.

The penultimate section of the disk contains inodes; the number of inodes is determined by the size of the disk.
The inodes have two representations: on-disk (\autoref{fig:inode on disk}) and in-memory (\autoref{fig:inode in memory}).
This allows the filesystem to make efficient use of disk space, while providing faster access to important values when the inode is loaded in memory.
There are 10 zones per inode: 7 direct zones (those that contain data), 1 single indirect zone (indicates a block that contains more direct zones), 1 double indirect zone (indicates a block that contains more single indirect zones), and 1 unused zone.
The unused zone is present for future use, potentially as a triple-indirect zone.

\begin{figure}[tb]
  \centering
  \begin{subfigure}[b]{0.49\textwidth}
    \centering
    \resizebox{0.8\textwidth}{!}{\incfig{inode-on-disk}}
    \caption{On disk}
    \label{fig:inode on disk}
  \end{subfigure}
  \hfill
  \begin{subfigure}[b]{0.49\textwidth}
    \centering
    \resizebox{0.7\textwidth}{!}{\incfig{inode-in-memory}}
    \caption{In memory}
    \label{fig:inode in memory}
  \end{subfigure}
  \hfill
  \caption{Inode representations}
  \label{fig:inode representations}
\end{figure}

The final section spans the rest of the blocks on the disk, broken into zones, and is available to store user data.
It can contain file data, or directory entries.

There are also some in-memory structures for working with open files.
The filesystem keeps a process table in memory, which is indexed by process ID (PID), and keeps track of inode information for all processes using the filesystem.
One entry corresponds to one process, and contains the inode numbers for the root and working directories of the process, as well as the list of open file descriptors.
Each open file descriptor corresponds to an entry in the filesystem's \textit{filp} table, which is shared among all processes and contains all the file position.
The rationale for a shared file position table comes from MINIX, and is based on problems with the semantics of the \textit{fork} system call \cite{tanenbaum1997}.
An entry in the filp table contains the number of file descriptors using that entry, the inode number, and the file position for the inode.

Due to time constraints, a number of simplifications were made compared to MINIX.
In particular, only the features necessary for basic functionality of the system were implemented, and only file operations are supported (create, read, write, delete).
The filesystem does not cache any information, and there is no in-memory inode table; all data are written directly to disk.
Furthermore, the current implementation does not keep track of file mode, owner, group, or timestamps.
This results in a limited filesystem, which nonetheless completes its function as a proof-of-concept.

  \section{Implementation}
After discussing the high-level design of the filesystem, we explore the details of the implementation, with particular focus on elements that are specific to Ada/SPARK.
The filesystem is implemented using the Ada 2012 programming language.
Some of its parts are written in the SPARK 2014 language, which is a subset of Ada that removes features not amenable to formal verification, and defines new aspects to support modular, constructive, formal verification \cite{sparkRM}.
We use the AdaCore GNAT Community 2020 package,\footnote{\url{https://www.adacore.com/community}} which provides, among others, the compiler and prover tools.
For testing purposes, a FUSE driver is written in C, and the built executables are linked with libfuse 3.9.2.\footnote{\url{https://github.com/libfuse/libfuse}}
The GNAT Project Manager is used to facilitate compilation of source files in different languages and linking of other required libraries.

\subsection{Overview}
The implementation of AdaFS has \result{loc} lines of code: \result{loc-specification} lines for the specification, and \result{loc-implementation} for the implementation.
\autoref{tab:lines of code} shows the number of lines of code in the entire project, separated by component.
The component that defines filesystem logic has the most lines of code in the specification, as it also contains the formal verification code, which is part of the specification.
The disk IO component contains the most lines of implementation code, as it takes care of the complexities of reading/writing data to disk.
The \textit{mkfs} utility does not contain any specification code, as it is not verified and it consists of a single main procedure (though it has nested procedures, those do not require a specification).

\begin{table}[tb]
  \centering
  \begin{tabular}{l | r | r}
    Component & Specification & Implementation \\
    \hline \hline
    Filesystem Logic & \result{loc-logic-specification} & \result{loc-logic-implementation} \\
    Disk IO & \result{loc-io-specification} & \result{loc-io-implementation} \\
    FUSE interaction & \result{loc-fuse-specification} & \result{loc-fuse-implementation} \\
    \textit{mkfs} utility & \result{loc-utility-specification} & \result{loc-utility-implementation}
  \end{tabular}
  \caption{Lines of code in AdaFS, separated by component.}
  \label{tab:lines of code}
\end{table}

\subsection{Language features}
Next, we outline some of Ada's features that have been particularly useful in the development of AdaFS, or that we consider unique and interesting.

\paragraph{Strong typing}
Ada is a strongly typed language, which helps the programmer distinguish between types that are logically different, even if their underlying representation is the same.
Furthermore, the compiler will automatically catch any bugs that would be caused by assigning a value of an incorrect type to a variable.
In order for a value to be assigned to a variable, two constraints must be satisfied: the value and variable must have the same type, and the value must satisfy all constraints on the variable (such as the range for an integral data type) \cite{barnes2014}.
Conversion between types is allowed, but only if it is explicit, and if the target type is an ancestor of the current type (for example, a positive integer may be converted to an integer, but not vice-versa).
An example of the use of types is the definition of a character buffer of arbitrary size, followed by the definitions of different block types including a constrained version of the buffer type, shown in \autoref{code:block type definitions}.

\begin{lstlisting}[float=tb,caption={Block type definitions}, label={code:block type definitions}, language=Ada]
type data_buf_t is array (Positive range <>) of Character;

type inode_block_t is array (1..block_size/on_disk'Size) of on_disk;
type zone_block_t is array (1..n_indirects_in_block) of Natural;
type dir_entry_block_t is array (1..block_size/direct'Size) of direct;
subtype data_block_t is data_buf_t (1..data_block_range'Last);
\end{lstlisting}

\paragraph{Controlled types}
Controlled types allow the programmer to specify precisely what happens when a variable of a given type is declared, or when it goes out of scope.
They are not permitted in SPARK, because the compiler inserts implicit calls to allow this functionality \cite{sparkRM}.
The initialization and destruction of variables of a controlled type is handled with custom \lstinline[language=Ada]{initialize} and \lstinline[language=Ada]{finalize} procedures for the type, thus the compiler needs to ensure these procedures are called in execution.
However, in AdaFS, controlled types are not used for any filesystem logic, and are only used with disk I/O to make sure that e.g. the variable containing the superblock is initialized properly with the type of I/O the filesystem uses.
Therefore, they were deemed acceptable for this prototype.

\paragraph{Lack of pointers}
The C codebase of MINIX makes extensive use of pointers to refer to inodes and other structures in memory, addresses on disk, etc.
Ada has access types, which are similar to pointers in some ways, but due to the complexity of verifying a program's behavior when it contains pointers, SPARK severely restricts the possibilities of using access types \cite{sparkRM}.
Thus, alternative methods must be used to provide similar functionality.

One way to solve this is with parameter modes.
Ada allows specifying the mode of each parameter in a procedure, which designates how the parameter will be used in execution.
If a parameter is of `in' mode, it is only read in the procedure, and is not modified -- this is the default mode.
If a parameter is of `out' mode, the value of the parameter before the call is irrelevant, as it will receive a value in the procedure.
Finally, if a parameter is of `in out' mode, it is both read and updated in the procedure.
This third mode provides functionality similar to a pointer.

However, SPARK requires that functions be purely functional; that is, they cannot have side effects, such as parameters with a mode of `out' or `in out'.
Here, two solutions are possible.
In some cases, it may be preferred to add the return value as an `out' parameter, and rewrite the function as a procedure.
In other cases, it is better for the function to return multiple values, which is possible with a record type.
An example is the function \lstinline[language=Ada]{parse_next} in \autoref{code:function returning record}, which returns two values, wrapped in the record type \lstinline[language=Ada]{parsed_res_t}.

A third solution is to simply use index values.
This was employed in the implementation of tables, such as the \textit{filp} table.
In MINIX, items in these tables are referred to using pointers, but AdaFS refers to them by index.

\begin{lstlisting}[float=tb,caption={Parse function returning the parsed component and the new cursor position (ellipses denote code omitted for brevity)}, label={code:function returning record}, language=Ada]
subtype cursor_t is Natural range path'Range;

type parsed_res_t is record
  next : adafs.name_t;
  new_cursor : cursor_t;
end record;

function parse_next
  (path : adafs.path_t; cursor : cursor_t) return parsed_res_t
is
  ...
end parse_next;
\end{lstlisting}

\paragraph{Modularisation}
Ada supports modularisation in the form of packages, and forms a single translation unit, which can contain member entities such as subprograms, variables, and types.
Information hiding is done by defining members as private.
A package is separated into two parts, which are placed in separate files: the \textit{specification} (the public interface for the package), and the \textit{body} (the implementation).
The compiler always checks whether the package body matches the specification, and refuses to compile the code if this is not the case.
\autoref{code:inode specification and body} shows an excerpt from the specification and body of the \lstinline[language=Ada]{adafs.inode} package.

A package can also have child packages: for example, the \lstinline[language=Ada]{adafs.inode} package is a child of the \lstinline[language=Ada]{adafs} package.
A child package has access to all member entities defined in the specification of its parent(s), including private members.

\begin{lstlisting}[float=tb,caption={Excerpt from the adafs.inode package specification and body (ellipses denote code omitted for brevity)}, label={code:inode specification and body}, language=Ada]
-- adafs-inode.ads
package adafs.inode
  with SPARK_Mode
is
  ...
  function calc_num_inodes_for_blocks (nblocks : Natural) return Natural
    with ...;
end package adafs.inode

-- adafs-inode.adb
package body adafs.inode
  with SPARK_Mode
is
  function calc_num_inodes_for_blocks
    (nblocks : Natural) return Natural
  is
    inode_max : constant := 65535;
    i : Natural := nblocks/3;
  begin
    ...
    return i;
  end calc_num_inodes_for_blocks;
end adafs.inode;
\end{lstlisting}

It is also possible to create \textit{generics}.
Generics are somewhat similar to objects in the Object-Oriented Programming paradigm, in that they can be instantiated with parameters.
However, an important difference is that instantiation can only occur in a declarative region (i.e. between the \lstinline[language=Ada]{is} and \lstinline[language=Ada]{begin} keywords).
Both subprograms and packages can be generic.
For example, \autoref{code:generic reading function} shows a generic function for reading data from a disk, and its instantiation.
A specification for the generic function \lstinline[language=Ada]{read_block} is defined in \textit{disk.ads}, accepting a numeric parameter between 0 and the disk size in blocks, and returning a result of the generic type \lstinline[language=Ada]{elem_t}.
In \textit{disk.adb}, the function is implemented.
If necessary, it first sets the disk file mode to read mode (\lstinline[language=Ada]{in_file}).
Then, it reads data of the generic type \lstinline[language=Ada]{elem_t} at the disk position corresponding to block number \lstinline[language=Ada]{num}, and returns these data.
The block-to-position conversion is handled by the \lstinline[language=Ada]{block2pos} function, which is omitted for brevity.
An example use of the generic function is in \textit{disk-inode.adb}, where the \lstinline[language=Ada]{read_block} function is instantiated with the data block type \lstinline[language=Ada]{data_block_t} (defined in another file to be an array of 1024 bytes).
The function instance is then used to read the data block \lstinline[language=Ada]{block_num}.

As Ada's \lstinline[language=Ada]{Stream_IO} always requires specifying the type to be read/written, using a generic function helps with code reuse.
\lstinline[language=Ada]{elem_t} is a generic type representing the type to be read or written; the concrete type is specified at instantiation.
The downside of generics is that they cannot be analyzed directly by SPARK, but must instead be verified from the context of instantiation (i.e., SPARK mode must be enabled in the package or subprogram that instantiates the generic) \cite{sparkRM}.

\begin{lstlisting}[float=tb,caption={Generic function for reading a block of type \textnormal{elem\_t}}, label={code:generic reading function}, language=Ada]
-- File: disk.ads
subtype block_num_t is Natural 0..disk_size_blocks;

generic
  type elem_t is private;
function read_block (num : block_num_t) return elem_t;

-- File: disk.adb
package SIO renames Ada.Streams.Stream_IO;
function read_block (num : block_num_t) return elem_t is
  result : elem_t;
begin
  if SIO.mode(disk) /= SIO.in_file then
    SIO.set_mode(disk, SIO.in_file);
  end if;

  SIO.set_index (disk, block2pos(num));
  elem_t'read (disk_access, result);
  return result;
end read_block;

-- File: disk-inode.adb
function read_chunk ... is
  function read_data_block is new read_block(data_block_t);
  data_block : data_block_t;
begin
  ...
  data_block := read_data_block(block_num);
  ...
end read_chunk;
\end{lstlisting}

\paragraph{Input \& output}
In AdaFS, we use a disk image file to represent the filesystem's disk.
Therefore, an appropriate way of reading and writing the file was needed.
Ada offers several types of input and output (IO), in the form of packages.
The simplest is \textit{Text\_IO}, which provides sequential file IO for human-readable text only.
The package \textit{Sequential\_IO} provides sequential access for heterogeneous data (data of varying types).
There are also two packages for random-access IO: \textit{Direct\_IO} and \textit{Stream\_IO}.
\textit{Direct\_IO} is used for files with homogeneous data (data of a single, uniform type), and \textit{Stream\_IO} is for heterogeneous data.

For this project, we required random-access IO, to allow the filesystem to read and write data at any position on the disk.
We also preferred heterogeneous IO, to be able to easily read and write data of varying types, such as different types of blocks (a data block, directory block, inode block, etc.).
Therefore, we opted for \textit{Stream\_IO}, as it best fulfilled our requirements.

\paragraph{Interfacing with other languages}
Ada has a mechanism to allow interfacing with other programming languages, such as Fortran, COBOL, or C.
This is done by replicating the types and subprogram signatures in Ada.
The Interfaces library package\footnote{\url{https://www.adaic.org/resources/add_content/standards/05aarm/html/AA-B-2.html}} provides types and subprograms for this purpose.
For example, for C, there are the Interfaces.C\footnote{\url{http://www.ada-auth.org/standards/12rm/html/RM-B-3.html}} and Interfaces.C.Strings\footnote{\url{http://www.ada-auth.org/standards/12rm/html/RM-B-3-1.html}} packages, which provide the types \lstinline[language=Ada]{chars_ptr} (mirroring \lstinline[language=C]{char*} in C), \lstinline[language=Ada]{int} (mirroring \lstinline[language=C]{int} in C), etc.
This allows \textit{exporting} subprograms from Ada, and calling them from a C program.
\autoref{code:interfacing c and ada} shows a specification of a subprogram that is exported to C by specifying the \lstinline[language=Ada]{Export} and \lstinline[language=Ada]{Convention} aspects (as well as an external name to use when calling the subprogram from C).
The subprogram is then declared as \lstinline[language=C]{extern} in C, and called from the C program's main function.
Since the main program is written in a language different from Ada, the initialization and finalization procedures (\lstinline[language=C]{adainit(void)} and \lstinline[language=C]{adafinal(void)}) must also be declared and called before and after any other Ada subprograms, respectively.
The GNAT Project Manager handles compiling and linking of files written in different languages.
Unfortunately, interfacing code is not amenable to formal verification.

\begin{lstlisting}[float=tb,caption={Interfacing code written in C and Ada. \textnormal{declarations.adb} is omitted for brevity, but is assumed to contain an implementation of the factorial function conforming to the specification.}, label={code:interfacing c and ada}, language=Ada, alsolanguage=C]
-- declarations.ads
with Interfaces.C;
package Declarations is
  function Factorial  (n : Interfaces.C.int) return Interfaces.C.int
    with Export => True,
         Convention => C,
         External_Name => "ada_factorial";
end Declarations;

-- main.c
#include <stdio.h>
extern void adainit (void);
extern void adafinal (void);
extern int ada_factorial(int n);

int main(int argc, const char *argv[]) {
  adainit();
  int n = 5;
  printf("%d\n", ada_factorial(n)); // 120
  adafinal();
}
\end{lstlisting}

\paragraph{FUSE \& the FUSE driver}
FUSE is a software interface that allows running filesystem code in userspace, with FUSE bridging the gap between the filesystem and the kernel.
This simplifies the development of filesystems, because access to the kernel and modification of kernel code is not necessary.
FUSE allows a filesystem to be developed iteratively; i.e., first it can be implemented and tested with FUSE, and later connected to a kernel if needed.

To implement a filesystem with FUSE, the code needs to be linked with the FUSE library (libfuse), and a driver must be written to specify the filesystem's handler functions for various operations.
As FUSE is written in C, the driver for AdaFS is currently also written in C, with a wrapper in Ada to convert values between C types and Ada types.
FUSE specifies a \lstinline[language=C]{struct} with pointers to functions that should be written by the programmer for the specific filesystem they are implemented.
The library also provides, among others, a function to fill file entries into a buffer, a \lstinline[language=C]{struct} to store open file information, and a function to get the context of the current operation (such as the PID requesting the operation).
\autoref{code:fuse open} shows an example from the driver, with an implementation of the open file operation.

FUSE also has support for multithreading and thus concurrent access.
However, this introduces a large amount of complexity to handle concurrent access in a filesystem.
Therefore, we run FUSE in single thread mode (with the \textit{-s} flag on the command line), which ensures that there can only be one ongoing file operation at a given time.

\begin{lstlisting}[float=tb,caption={FUSE driver implementation of \textnormal{open}}, label={code:fuse open}, language=C]
#define FUSE_USE_VERSION 31
#include <fuse.h>
// Declare the external filesystem open function written in Ada
extern int ada_open(const char *path, pid_t pid);

// The driver's open function
int adafs_open(const char *path, struct fuse_file_info *finfo) {
  pid_t pid = fuse_get_context()->pid;
  int fd = ada_open(path, pid);
  finfo->fh = fd;
  return 0;
}

// Register the function with FUSE
static struct fuse_operations adafs_ops = {
  .open = adafs_open
};

int main(int argc, char **argv) {
  ...
  return fuse_main(argc, argv, &adafs_ops, NULL);
}
\end{lstlisting}

\paragraph{Formal verification}
Unfortunately, much of the code in its current form is not amenable to formal verification.
These are namely the parts involving file input and output, and functions that work with C types (i.e. the FUSE driver).
To mitigate this, we attempted to make use of an existing project containing FUSE bindings for Ada,\footnote{\url{https://github.com/medsec/ada-fuse}} but as the project had not been maintained since 2016, we were unable to compile it.

Nevertheless, large parts of filesystem logic are formally verifiable.
Therefore, the code base was split into two distinct packages: the \textit{adafs} package, which contains filesystem logic that is strictly in the SPARK language (with the exceptions of \textit{adafs.operations}), and the \textit{disk} package, which contains unverifiable elements such as disk I/O.
The \textit{adafs} package does not include any specifics about the type of disk being used, as the \textit{disk} interface hides implementation details.
Thus, when needed, and when an alternative is found, the \textit{disk} package can simply be replaced with a verifiable implementation that exposes an identical interface.

For the parts that are formally verifiable, two types of contracts are available: functional and data contracts.
Functional contracts describe how a subprogram should function; that is, the pre- and post-conditions for a given subprogram.
They are written as boolean predicate logic expressions.
Pre-conditions are evaluated before entry into the subprogram, and post-conditions are evaluated after exit from the subprogram (and can therefore mention the subprogram's result).
SPARK can check these conditions at each call site to ensure that no subprogram call violates the conditions, and that the output(s) are shown to be conformant with the specification (the returned result for a function, and the \lstinline[language=Ada]{out} or \lstinline[language=Ada]{in out} parameters for a procedure).

The second type of contract available are data contracts.
SPARK conducts flow analysis, which models the flow of information during a subprogram's execution.
It checks for uninitialized variables, ineffective statements, and incorrect parameter modes.
It is possible to specify which global variables are read, written, or both read and written in the subprogram, using the \lstinline[language=Ada]{Global} aspect.
If no global variables are used, the value of the aspect is set to \lstinline[language=Ada]{null}.
It is also possible to specify data dependencies between a subprogram's inputs and outputs.

For example, \autoref{code:formal verification example} shows a function to get an entry from the process table, which specifies the functional and data contracts to be fulfilled for the function.
The \lstinline[language=Ada]{Global} aspect specifies that the function only depends on the value of the package variable \lstinline[language=Ada]{tab} for input.
The \lstinline[language=Ada]{Depends} aspect states that the result of the function only depends on the \lstinline[language=Ada]{tab} and \lstinline[language=Ada]{pid} variables (that is, the variable stated in the \lstinline[language=Ada]{Global} aspect, and the parameter of the function).
The post-condition states that the \lstinline[language=Ada]{is_null} component of the returned variant record will be set to True if there is no entry in \lstinline[language=Ada]{tab} for the provided PID; otherwise, the inode number of the PIDs working directory will be non-zero.
With the SPARK toolchain, we can verify that these constraints are all satisfied.
Verification is fully automatic; it does not require any human intervention.

\begin{lstlisting}[float=tb,caption={Functional and data contracts}, label={code:formal verification example}, language=Ada]
function get_entry (pid : tab_range) return entry_t with
  Global => (input => tab),
  Depends => (get_entry'Result => (tab, pid)),
  Post => (if tab(pid).is_null
           then get_entry'Result.is_null
           else get_entry'Result.workdir > 0);
\end{lstlisting}

  \section{Results \& evaluation}
As stated earlier, the filesystem was split into two parts: a filesystem logic part (the \textit{adafs} package), and an I/O part (the \textit{disk} package).
Because the logic part is easier to verify, it was the primary focus of formal verification for this project.
Subprograms were given functional and data contracts, and type definitions according to their expected functionality.
Although this does not formally verify the functionality of the entire filesystem, it is a good starting point.
\autoref{code:prover summary} shows a summary of the conducted checks, and \autoref{code:prover output} shows some of the locations of various checks.

\begin{lstlisting}[basicstyle=\tiny, caption={Prover summary (\textnormal{gnatprove})}, label={code:prover summary}]
Summary of SPARK analysis
=========================

---------------------------------------------------------------------------------------------------------
SPARK Analysis results        Total        Flow   Interval   CodePeer      Provers   Justified   Unproved
---------------------------------------------------------------------------------------------------------
Data Dependencies                 5           5          .          .            .           .          .
Flow Dependencies                 4           4          .          .            .           .          .
Initialization                    2           2          .          .            .           .          .
Non-Aliasing                      .           .          .          .            .           .          .
Run-time Checks                  19           .          .          .    19 (CVC4)           .          .
Assertions                        2           .          .          .     2 (CVC4)           .          .
Functional Contracts              6           .          .          .     6 (CVC4)           .          .
LSP Verification                  .           .          .          .            .           .          .
---------------------------------------------------------------------------------------------------------
Total                            38    11 (29%)          .          .     27 (71%)           .          .

max steps used for successful proof: 1

Analyzed 18 units
\end{lstlisting}

\begin{lstlisting}[basicstyle=\tiny, caption={Excerpt from the prover output (\textnormal{gnatprove})}, label={code:prover output}]
adafs-inode.adb:6:27: info: division check proved (CVC4: 1 VC in max 0.0 seconds and 1 step)
adafs-inode.adb:6:27: info: range check proved (CVC4: 1 VC in max 0.0 seconds and 1 step)
adafs-inode.adb:8:44: info: division check proved (CVC4: 1 VC in max 0.0 seconds and 1 step)
adafs-inode.adb:13:12: info: overflow check proved (CVC4: 1 VC in max 0.0 seconds and 1 step)
adafs-inode.adb:14:12: info: division check proved (CVC4: 1 VC in max 0.0 seconds and 1 step)
adafs-inode.adb:14:12: info: range check proved (CVC4: 1 VC in max 0.0 seconds and 1 step)
adafs-inode.adb:14:28: info: overflow check proved (CVC4: 1 VC in max 0.0 seconds and 1 step)
adafs-inode.ads:40:60: info: range check proved (CVC4: 1 VC in max 0.0 seconds and 1 step)
adafs-inode.ads:66:10: info: data dependencies proved
adafs-inode.ads:67:10: info: flow dependencies proved
adafs-inode.ads:69:18: info: postcondition proved (CVC4: 2 VC in max 0.0 seconds and 1 step)
adafs-proc.adb:6:14: info: discriminant check proved (CVC4: 2 VC in max 0.0 seconds and 1 step)
adafs-proc.adb:12:16: info: discriminant check proved (CVC4: 2 VC in max 0.0 seconds and 1 step)
adafs-proc.ads:26:05: info: data dependencies proved
adafs-filp.adb:18:30: info: loop invariant initialization proved (CVC4: 1 VC in max 0.0 seconds and 1 step)
adafs-filp.adb:18:30: info: loop invariant preservation proved (CVC4: 1 VC in max 0.0 seconds and 1 step)
adafs-filp.ads:28:13: info: postcondition proved (CVC4: 3 VC in max 0.0 seconds and 1 step)
adafs-filp.ads:32:28: info: initialization of "free_fd" proved
adafs-filp.ads:33:05: info: data dependencies proved
adafs-filp.ads:34:05: info: flow dependencies proved
adafs-filp.ads:35:13: info: postcondition proved (CVC4: 2 VC in max 0.0 seconds and 1 step)
\end{lstlisting}

The checks done by the prover can each be related directly to a Common Weakness Enumeration (CWE) number; these are shown in \autoref{tab:checks and cwe numbers} \cite{sparkUG}.

\begin{table}[h]
  \centering
  \vspace{1em}
  \renewcommand{\arraystretch}{1.5}
  \begin{tabular}{| l | l |}
    \hline
    \textbf{Message} & \textbf{CWE number} \\ \hline
    divide by zero & 369 \\ \hline
    index check & 120 \\ \hline
    overflow check & 190 \\ \hline
    fp\_overflow check & 739 \\ \hline
    range check & 682 \\ \hline
    predicate check & 682 \\ \hline
    predicate check on default value & 682 \\ \hline
    null pointer dereference & 476 \\ \hline
    memory leak & 401 \\ \hline
    memory leak at end of scope & 401 \\ \hline
    discriminant check & 136 \\ \hline
    tag check & 136 \\ \hline
    use of an uninitialized variable & 457 \\ \hline
    precondition & 628 \\ \hline
    postcondition & 682 \\ \hline
  \end{tabular}
  \renewcommand{\arraystretch}{1}
  \vspace{1em}
  \caption{SPARK checks with their respective CWE numbers}
  \label{tab:checks and cwe numbers}
\end{table}

Due to Ada's type system, many of the checks do not take much effort to verify; in fact, all of the conducted checks were verified in one step.
The assurances provided by the prover are powerful, as they mean that e.g. overflow errors and range errors can \textit{never} happen at those locations.
In standard Ada, instructions are emitted to check that e.g. array accesses are within the bounds of the array.
However, with SPARK, such verifications can be conducted ahead of time.
This means that the runtime checks done by Ada can be safely disabled, eliminating a number of machine instructions and thus improving the performance of the system.

We compare the performance of AdaFS with a C implementation of a FUSE-based MINIX 3 filesystem\footnote{\url{https://github.com/redcap97/fuse-mfs}}.
As AdaFS is not a complete filesystem, but only supports a small set of operations, it was not possible to use a standardised utility to test it.
Therefore, we created a Bash script that uses the \textit{time} built-in command to measure the time a command takes to execute, and used the commands \textit{touch}, \textit{echo}, and \textit{rm} to work with files on the filesystem.
\autoref{tab:fs times} shows the comparisons of some common filesystem operations, such as creating, writing, reading, and deleting files.
The \textit{time} command outputs three different time values: Real, User, and Sys.
Real is the ``wall clock'' time, User is the amount of time that the CPU spent on the program in user mode, and Sys is the amount of time that the CPU spent in kernel mode during the execution of the program.
We are particularly interested in the User and Sys values, as the Real value may be affected by other programs running at the same time.
The results show that although overall, the Ada implementation is technically slower, there is no significant performance difference between the C and Ada programs
In particular, formal verification does not introduce any noticeable latency.
This is also because the checks can be verified at compile time by the prover, which means that they do not need to be conducted at runtime.

% comparing with this: https://github.com/redcap97/fuse-mfs
\begin{table}[h]
  \begin{subtable}[t]{0.45\textwidth}
    \centering
    \begin{tabular}{l | l | l}
      Category & C & Ada \\
      \hline \hline
      Real & 47 & 103 \\
      User & 26 & 32 \\
      Sys & 13 & 9
    \end{tabular}
    \caption{Time to create 30 files (in milliseconds)}
    \label{tab:create 30}
  \end{subtable}
  \hfill
  \begin{subtable}[t]{0.45\textwidth}
    \centering
    \begin{tabular}{l | l | l}
      Category & C & Ada \\
      \hline \hline
      Real & 38 & 51 \\
      User & 22 & 23 \\
      Sys & 12 & 10
    \end{tabular}
    \caption{Time to delete 30 files (in milliseconds)}
    \label{tab:delete 30}
  \end{subtable}

  \bigskip

  \begin{subtable}[t]{0.45\textwidth}
    \centering
    \begin{tabular}{l | l | l}
      Category & C & Ada \\
      \hline \hline
      Real & 3 & 6 \\
      User & 2 & 2 \\
      Sys & 0 & 0
    \end{tabular}
    \caption{Time to create and delete a file (in milliseconds)}
    \label{tab:create delete}
  \end{subtable}
  \hfill
  \begin{subtable}[t]{0.45\textwidth}
    \centering
    \begin{tabular}{l | l | l}
      Category & C & Ada \\
      \hline \hline
      Real & 2 & 5 \\
      User & 2 & 1 \\
      Sys & 0 & 1
    \end{tabular}
    \caption{Time to write and read data (in milliseconds)}
    \label{tab:write read}
  \end{subtable}

  \caption{Comparing filesystem operation performance in C and Ada. (Real = wall clock time, User = user mode time, Sys = kernel mode time)}
  \label{tab:fs times}
\end{table}

  \section{Related Work} \label{sec:related work}
Having presented our design and implementation, we relate it to the existing body of work.
Other research has mainly focused on two areas: tools or frameworks for writing reliable filesystem code, and development of reliable filesystems themselves.

\subsection{Tools \& Frameworks}\label{subsec:tools frameworks}
Fryer et al. found that filesystem bugs that severely corrupt metadata are common, and that solutions to the necessary recovery procedures were unsatisfactory.
Therefore, they developed Recon, a system to protect filesystem metadata from arbitrary implementation bugs \cite{fryer2012}.
Recon sits between the filesystem and the block layer, and checks consistency invariants at commit points before allowing writes to disk.
Thus, failures that would be silent become detectable violations of these invariants.
Consistency invariants are declarative statements that must be satisfied before data is committed to disk, otherwise the filesystem may become corrupted.
The authors use the consistency rules used by a file system checker (\textit{e2fsck} in their implementation) to derive the invariants.
Consequently, the system can detect random corruption at runtime as effectively as a filesystem checker.
In essence, Recon can conduct online checks similar to those conducted by offline filesystem checkers (e.g. \textit{fsck}).
The disadvantage is that Recon is limited only to errors detectable by filesystem checkers, and it does not allow the programmer to specify which conditions should be checked.
It also only detects errors in filesystem metadata, not in user data.
Nevertheless, it is a general-purpose tool, and functions identically regardless of the implementation of various layers.

Another tool is COGENT, a new, restricted language developed by Amani et al. \cite{amani2016}, as an approach to writing and formally verifying high-assurance filesystem code.
It was designed to bridge the gap between the formal specification of a program and its low-level implementation, and to allow programmers that do not have a background in formal verification to provably avoid common errors.
Many of the characteristics required to guarantee the absence of common filesystem implementation errors were encoded in the language, which is strongly typed, type safe, and uses a linear type system (meaning that every variable must be used exactly once).
The compiler for the language generates C code and translation correctness proofs, enforces memory safety, and forbids undefined behavior in C code (such as null pointer dereferences or buffer overflows).
In evaluation, they found that code generated from COGENT has throughput that is almost identical to a C implementation, albeit with slightly higher CPU usage.
Though developing a language that has the necessary constraints encoded in its syntax and semantics may be a useful approach, the downside is that COGENT is very domain-specific.
This means that programmers would need to first learn the language, which may be more difficult given that it is a purely functional language whose type system is not very common, and people developing systems code are arguably more used to working with imperative languages.
However, it is important to note that the authors state that this was not a major barrier.

In 2016, Sigurbjarnarson et al. presented Yggdrasil, a toolkit which uses a push-button approach to formal verification \cite{sigurbjarnarson2016}.
Yggdrasil does not require manual annotations or proofs, and aids programmers by producing counterexamples for failed verification constraints.
To achieve this, the authors used \textit{crash refinement}, i.e. that the set of all disk states reachable in the implementation must be a subset of those allowed in the specification.
Push-button verification means that Yggdrasil asks the programmer to enter the specification of the expected behavior, the implementation, and any consistency invariants for the state of the filesystem image, and it then checks if the implementation meets the specification while satisfying the invariants.
All three inputs are written in the same language (a subset of Python), and Yggdrasil generates C code, which is compiled to an executable filesystem and can be mounted using the FUSE library.
The advantage is that proof is fully automatic, and does not require any special annotations in the code.
There is, however, the question of whether a high-level language such as Python provides access to all low-level functionality that a filesystem programmer may need, without importing modules (as it is unclear how imports are handled, and whether they are allowed).

Argosy is a framework to allow machine-checked verification of storage systems, and introduces recovery refinement, which is a set of conditions that guarantee that an implementation of an interface with a recovery procedure is correct \cite{chajed2019}.
Recovery refinement ensures correctness for anything using the specification, and can compose with other refinements to prove that an entire system is correct.
Its semantics are formulated in Kleene algebra.
The system implements Crash Hoare logic, which was introduced by and used in FSCQ, to prove recovery refinement.
Therefore, the authors produce a verified transactional disk API for unreliable disks.
Similarly to FSCQ, the code is verified using the Coq proof assistant, and produces Haskell code.

\subsection{Filesystems}
In 2014, Schellhorn et al. presented their work on a verified Flash filesystem \cite{schellhorn2014}.
Since Flash memory can only be written sequentially, and data in Flash memory cannot be overwritten in place (i.e. space can only be reused by erasing entire blocks), a special Flash file system must be used that is designed to work with Flash memory.
Flashix, the verified Flash filesystem they developed, is based on UBIFS.
They refined the high-level POSIX system interface using a Virtual Filesystem Switch (VFS) that maps POSIX operations to one or more Abstract File System (AFS) operations; the AFS is a model that captures the functional behavior of a specific filesystem \cite{ernst2012}.
The reason for this separation is that specific filesystems do not implement generic functionality, but instead satisfy an internal interface.
To provide safety for crashes, power cuts, and other such events, they used a transactional journal providing atomic writes \cite{ernst2015}.
They verified the code using the KIV interactive theorem prover, whose specification language is based on Abstract State Machines and Abstract State Machine refinement.
They then used tools to generate a Scala implementation, which could be mounted using the FUSE library and executed on the Java Virtual Machine.

FSCQ was the first file system that has a machine-checkable proof (as opposed to an interactive proof) that its specification and implementation match, and whose specification included crashes \cite{chen2015specifying}.
Chen et al. found that to achieve their design goals of atomic system calls, preventing real bugs, enabling proof automation, and allowing modularity, an extended variant of Hoare logic worked best \cite{chen2015using}.
Accordingly, they developed Crash Hoare logic, which allows programmers to write a specification of the behavior of a storage system in the face of a crash, and to prove them correct (i.e. if a computer crashes, the storage system will reboot into a state consistent with its specification).
This extension of Hoare logic takes into account the fact that during a crash, a procedure may stop at any point, and that recovery procedures may run.
The filesystem uses FscqLog, a write-ahead log also certified with Crash Hoare logic, which provides atomic transactions on top of asynchronous disk writes.
FSCQ was developed with the same features as the educational xv6 filesystem, which implements the Unix v6 filesystem with write-ahead logging.
The implementation used the Coq proof assistant, and generated Haskell code, which could be mounted with the FUSE library and a Haskell driver.
Based on FSCQ, DFSCQ (Deferred-write FSCQ) was later written to provide a precise specification for \textit{fsync} and \textit{fdatasync} in the case of log-bypass writes \cite{chen2017}.
Deferring writing data to persistent storage allows the filesystem to achieve high I/O performance, and DFSCQ's implementation would provide crash safety for these operations.
In building DFSCQ, the authors presented a tree-based approach to specifying filesystem behavior, and a metadata-prefix specification to specify behavior for crashes.
Compared to FSCQ, DFSCQ has several optimisations and provides a number of missing features.
The approach of Chen et al. means that even the complex semantics of crashes and recovery procedures can be captured and verified.
However, it also means that a programmer wishing to use this system needs to be familiar with Coq's programming language, which is functional and dependently typed; this may be difficult for developers used to C-style languages.
Also, the resulting implementation is in Haskell, which may introduce overhead not present in C implementations.

Sigurbjarnarson et al. used Yggdrasil, discussed in \autoref{subsec:tools frameworks}, to implement a journaling filesystem: Yxv6.
This filesystem is similar to xv6 and FSCQ, with some differences.
To ensure correct atomicity of some operations, it manages opened files that have been unlinked using a partition for orphan inodes.
Unlike FSCQ, it uses validation, not verification, when managing free inodes and blocks; therefore, allocation of blocks or inodes may fail even if there is enough space.
The authors also implemented a verified version of the \textit{cp} utility, Ycp, and a verified implementation of the Arrakis \cite{peter2014} persistent log (Ylog).

  \section{Conclusion}
Summary and concluding remarks, including possible future work.

  \printbibliography

\end{document}
